
\documentclass[12pt, a4paper]{article}

\usepackage[ruled, vlined]{algorithm2e}

\author{Hong Lu \\ luh.lewis@gmail.com}
\date{}

\begin{document}
    \title{Supposed Framework for Task Arrangement and Execution}
    \maketitle

    \section{Reviews}
    \subsection*{How to deal with the tasks?}
    We can adapt several representations to the tasks by using various algorithms based on different data structures.
    The idea, we often bring about easily, is to connect or divide the tasks into an entity or several groups.
    \subsubsection*{k-means}
    Within clustering algorithm, k-means is efficient and gurantees to converge.
    \begin{algorithm}[H]
        \SetAlgoLined

        \KwData{k, Points, initial centroids}  
        \KwResult{Clusters: C}

        \caption{Jejune K-means}
    \end{algorithm}
\end{document}